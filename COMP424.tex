\documentclass[10pt, a4paper, oneside]{book}

\usepackage{titlesec}
\usepackage{lipsum}
\usepackage{color}
\usepackage{mathtools}
\DeclarePairedDelimiter\ceil{\lceil}{\rceil}
\DeclarePairedDelimiter\floor{\lfloor}{\rfloor}

\usepackage{amsthm}
\usepackage{booktabs}
\usepackage[table,xcdraw]{xcolor}

\usepackage{geometry}
\geometry{left=2cm,right=2cm,top=1.5cm,bottom=1.5cm}

\usepackage{array}
\usepackage{graphics}
\usepackage{amsmath}
\usepackage{color}   %May be necessary if you want to color links
\usepackage{hyperref}
\hypersetup{
    colorlinks=true, %set true if you want colored links
    linktoc=all,     %set to all if you want both sections and subsections linked
    linkcolor=black,  %choose some color if you want links to stand out
}

\usepackage{array}

\newcolumntype{M}[1]{>{\centering\arraybackslash}m{#1}}
\newcolumntype{N}{@{}m{0pt}@{}}

\usepackage[T1]{fontenc}
\usepackage{titlesec, blindtext, color}
\definecolor{gray75}{gray}{0.75}
\newcommand{\hsp}{\hspace{20pt}}
\titleformat{\chapter}[hang]{\Huge\bfseries}{\thechapter\hsp\textcolor{gray75}{|}\hsp}{0pt}{\Huge\bfseries}
\titlespacing{\chapter}{0pt}{\parskip}{*0.5}

\newtheoremstyle{theoremdd}% name of the style to be used
  {\topsep}% measure of space to leave above the theorem. E.g.: 3pt
  {\topsep}% measure of space to leave below the theorem. E.g.: 3pt
  {\itshape}% name of font to use in the body of the theorem
  {10pt}% measure of space to indent
  {\bfseries}% name of head font
  {. ---}% punctuation between head and body
  { }% space after theorem head; " " = normal interword space
  {}

\theoremstyle{theoremdd}
\newtheorem*{definition}{Definition}

\newtheorem{theorem}{Theorem}[chapter]

\newtheorem{corollary}{Corollary}

\newtheorem{example}{Example}[chapter]

\usepackage{tikz}
\newcommand*\circled[1]{\tikz[baseline=(char.base)]{
            \node[shape=circle,draw,inner sep=2pt] (char) {#1};}}

\usepackage[listings]{tcolorbox}
\tcbuselibrary{listings,theorems}

\newtcbtheorem[number within=section]{mytheo}{Theorem}%
{colback=white,colframe=black!,fonttitle=\bfseries}{th}

\usepackage[ruled,vlined,algosection]{algorithm2e}

\theoremstyle{remark}
\newtheorem*{remark}{Remark}

\newtheorem{lemma}{Lemma}[section]

\author{Wenzong Xia}
\title{COMP 424 Artificial Intelligence - Winter 2019}
\begin{document}
\maketitle
\tableofcontents

\part{Introduction}
\section*{Biological Intelligence}

What is biological intelligence? A mix of general-purpose and special-purpose algorithms. \textbf{General purpose:} Memory formation, updating, retrieval. Learning new tasks. \textbf{Special-purpose:} Recognizing visual patterns. Recognizing sounds. Learning language. All are integrated seamlessly.

\section*{Definition of AI}
\textbf{Working definition of AI:} Developing models and algorithms that can produce rational behaviors in response to incoming stimulus and information.
\begin{figure}[h!]
\includegraphics[width=100mm]{./img/AI_def.png}
\end{figure}

\section*{Goals of AI}
The general goals of AI: 
\begin{center}
\renewcommand{\arraystretch}{3}
\begin{tabular}{ c  c  c }
Thinking Humanly \cellcolor{yellow} & Thinking Rationally \cellcolor{yellow} & $\longleftarrow $ Hard to keep track of the progress \\ 
Acting Humanly   & Acting Rationally \cellcolor{green} & \\ 
                 & \put(50, 10){\vector(0,1){20}} The focus of the course & \\
\end{tabular}
\end{center}

\subsection*{Acting Rationally}
\begin{itemize}
\item Rational behavior = doing the 'right' thing.
\item Doing what is expected to \textbf{maximize goal achievement}, given the \textbf{available information} and \textbf{available resources}.
\begin{itemize}
\item Does not necessarily require thinking (e.g. blinking reflex).
\item But in many cases, thinking serves rational behavior.
\item This is the flavor of AI we will focus on.
\end{itemize}
\end{itemize}

\subsection*{Rational Agents}
\begin{itemize}
\item This course is about designing \textbf{rational agents}.
\begin{itemize}
\item An agent is an entity that perceives and acts.
\item Goal: Learn a function mapping percept histories to actions: $f: p^{h} \rightarrow A$
\end{itemize}
\item A rational agent implements this function such as to maximize performance.
\item \textbf{Caveat}: resources constraints (time, space, energy, bandwidth,...) which make perfect rationality unachievable.
\item \textbf{Objective}: Find best function for given information and resources.
\end{itemize}

\part{Basic Tools}
\chapter{Search}
\section{Uninformed Search}
Search is at the heart of all AI systems. Typical setup of an AI task includes:
\begin{itemize}
\item \textbf{Knowledge}: 
\begin{itemize}
\item Formal representation of the problem to be solved.
\item Includes definitions of state, goal, available actions.
\item Other factual or probabilistic knowledge about the problem.
\end{itemize}
\end{itemize}
\begin{itemize}
\item \textbf{Search}:
\begin{itemize}
\item Process of finding a solution (or possibly, the \textbf{best} solution).
\end{itemize}
\end{itemize}

\subsection{Defining a Search Problem}
Before we solve a problem, we need to formulate the problem into a search problem:
\begin{itemize}
\item \textbf{State space $S$}: all possible configurations of the domain.
\item \textbf{Initial state $s_{0} \in S$}: the start state. 
\item \textbf{Goal states $G \subset S$}: the set of end states.
\item \textbf{Operators $A$}: the actions available (often defined in terms of mapping from state to successor state).
\item \textbf{Path}: a sequence of states and operators.
\item \textbf{Path cost $C$}: a number associated with any path.
\item \textbf{Solution}: a path from $s_{0}$ to $s_{g} \in G$.
\item \textbf{Optimal solution}: a path with minimum cost.
\end{itemize}

\subsection{Basic Assumptions}
For next few lectures, simplifying assumptions for now. Later, we will consider search in other settings.
\begin{itemize}
\item \textbf{Static:} The environment will not change in the course of the search. vs \textbf{dynamic} environment.
\item \textbf{Observable:} The agent always know the current state.  vs \textbf{unobservable} environment. 
\item \textbf{Discrete:} At any given states, there are only finite many actions to choose from. vs \textbf{continuous} states.
\item \textbf{Deterministic:} Each action has exactly one outcome. vs \textbf{stochastic} environment.
\end{itemize}
Under these assumptions, the solution to any problem is a fixed sequence of actions.

\subsubsection*{Eight Puzzle Example}

\subsection{Search Graph vs Search Tree}
The process of looking for a sequence of actions that reach the goal is called \textbf{search}. A search algorithm takes a problem as input and returns a solution in the form of an action sequence. Once a solution is found, the actions it recommended can be carried out and thus in a execution phase.
\subsubsection*{Search Graph}
Search graph can visualize the state space search in terms of a graph. Graph defined by a set of vertices and a set of edges connecting the vertices. Nodes correspond to states in $S$. Edges correspond to operators. 
\subsubsection*{Search Tree}
Search tree represents the exploration paths in a search procedure. Nodes represent partial solutions, including a state in $S$. Edges correspond to operators. 

\subsection{Data Structure for Search Tree}
Defining a search tree node: 
\begin{itemize}
\item Each node contains a state id (from states in the graph).
\item Nodes also contain additional information: 
\begin{itemize}
\item The \underline{parent state} and the \underline{operator} used to generate it.
\item \underline{Cost} of the path \underline{so far}.
\item \underline{Depth} of the node.
\end{itemize}
\end{itemize}
Expanding a search tree node: 
\begin{itemize}
\item Applying \underline{all} legal operators to the state.
\item Generating nodes for all the corresponding successor states.
\end{itemize}

\subsection{Generic Search Algorithm}
\begin{algorithm}[H]
\SetAlgoLined
\caption{Generic Search Algorithm}
\KwResult{A solution/Failure}
\textbf{Initialize} the search tree using the initial state of the problem\;
\textbf{Repeat} \\
\uIf{No candidate nodes can be expanded}{return \textcolor{red}{Failure}}
Choose a node for expansion, according to some search strategy\;
\eIf{The node contains a \textcolor{blue}{goal state}}{return the corresponding path}{Expand the node by applying each applicable \textcolor{blue}{operator}, generating the \textcolor{blue}{successor state} and adding the resulting node to the tree}
\end{algorithm}



\part{Logical Representations}
\chapter{Game Playing}
\chapter{Logical Reasoning}
\chapter{Classical Planning}

\part{Probabilistic Representations}
\chapter{Probabilistic Reasoning}
\chapter{Learning Probabilistic Models}

\part{Utility Theory}
\chapter{Reasoning With Utilities}
\chapter{Sequential Reasoning And Decision-making}
\chapter{Learning Complex Sequential Decisions}
\chapter{Applications}









\end{document}